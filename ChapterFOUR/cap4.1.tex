\section{Sequências de Vídeo}
\label{cap:4.1}

A literatura científica na área de codificação de vídeo descreve a utilização de diversas sequências de vídeo em experimentos, desde resoluções muito baixas, como \textit{Quarter Common Intermediate Format} (QCIF, de 176$\times$144 pixels), até ultra-alta resolução, como \textit{Ultra-High Definition} 8K (UHD8K, de 7680$\times$4320 pixels), além das sequências de vídeo não tradicionais, como as de captura de tela (\textit{Screen Content Coding}, SCC), de alta profundidade de bits (\textit{High Dynamic Range}, HDR), vídeos omnidirecionais (também chamados de vídeos 360$^{\circ}$) e sequências de vídeo multivistas usadas em aplicações 3D e \textit{Light Fields}. Portanto, ao se realizar qualquer tipo de experimento na área de codificação de vídeo, é necessário estipular as sequências de vídeo que farão parte dele, pois essas escolhas impactam diretamente nos resultados a serem apresentados. Os diversos documentos que regram as condições ideais de teste, como \citet{bib:hevcctc} para o H.265/HEVC, o \citet{bib:vvcctc} para o H.266/VVC, o \citet{bib:ietfnetvct} para os formatos Daala, VP9 e AV1 e, por fim, o \citet{bib:av2ctc} para o futuro codificador AV2 \cite{bib:av2_avm}, ainda em desenvolvimento, tendem a dividir as diversas opções de sequências em categorias, cada uma voltada para um objetivo-fim.

Desta forma, apresentamos no Apêndice \ref{apx:A} todas as sequências utilizadas em diversos momentos nesta tese, classificando-os conforme duas características principais: resolução principal do vídeo e se é ou não uma sequência SCC. Portanto, ao longo desta tese foram utilizados 78 sequências de vídeos distribuídos em seis categorias: \textit{Common Intermediate Format} (CIF, de 426$\times$240 pixeis), \textit{Standard Definition} (SD, de 640$\times$360 pixeis), \textit{High Definition} 720p (HD720, de 1280$\times$720 pixeis), \textit{High Definition} 1080p (HD1080, de 1920$\times$1080 pixeis), HD1080 com \textit{Screen Content Coding} (HD1080+SCC) e \textit{Ultra High Definition} 4K (UHD4K, de 4096$\times$2160 pixeis). No Apêndice \ref{apx:A} também é possível observar informações gerais sobre o vídeo, tais como profundidade de bits, quadros por unidade de tempo (cuja divisão retorna a o valor de quadros por segundo), informações sobre informação espacial e informação temporal (respectivamente do inglês, \textit{Spatial Information} e \textit{Temporal Information}). Também incluímos uma breve descrição do que ocorre no vídeo e as seções e subseções em que essas sequências são utilizadas. Todas as sequências utilizadas nesta tese possuem subamostragem de pixels de 4:2:0, e todos os testes sempre utilizam os primeiros 60 quadros de cada sequência.
