\chapter{Conclusão}
\label{cap:conclusao}

No decorrer desta tese, foram apresentadas a importância e a relevância do uso de codificadores de vídeo, em especial quando falamos em transmissões online de vídeos. Ao longo da história, diversos formatos de codificação de vídeo foram propostos, sendo que o de maior destaque no âmbito de \textit{streaming} é o H.264/AVC que ainda hoje em dia é largamente utilizado. Além deste formato, outros três formatos de codificação de vídeo são amplamente utilizados pela indústria de \textit{streaming} nos últimos anos: H.265/HEVC, VP9 e AV1. Recentemente foi publicado o formato H.266/VVC, atual estado da arte em termos de codificação de vídeo. Desses formatos, apenas o AV1 é livre de royalties e foi especialmente projetado para uso em serviços de transmissão pela internet e com alta capacidade de compressão de dados. Sendo assim, esta tese focou em soluções para o formato AV1.

Esta tese abordou a necessidade de realizar modificações nos vídeos codificados, mais especificamente através da transcodificação de vídeo, a fim de adequar os \textit{bitstreams} para as mais diversas condições do usuário. Há diversos estudos, sejam eles acadêmicos ou não, sobre a utilização de transcodificadores de vídeo. Os principais trabalhos acadêmicos visam a aceleração do processo, haja vista a complexidade de execução de um codificador de vídeo. As soluções existentes reduzem a complexidade de execução do transcodificador, buscando não comprometer significativamente a eficiência de codificação.

Esta tese apresentou  uma revisão sistemática da literatura e uma comparação entre trabalhos envolvendo transcodificação de vídeo desenvolvidos a partir do ano 2011 até o ano 2022. Muitas observações puderam ser feitas sobre esse levantamento bibliográfico, das quais destacamos as seguintes:

\begin{itemize}
    \item 70,69\% das transcodificações rápidas basearam as suas propostas de aceleração no reaproveitamento de informações referentes a estruturas de particionamento de blocos;
    
    \item A média de aceleração obtida (TS) com as propostas de transcodificador rápido foi de 50,74\%, havendo um desvio padrão de 19,23\% do tempo de execução em relação ao transcodificador original;

    \item O impacto de perda em eficiência de codificação tende a  4,11\%, podendo variar entre mais ou menos 2,46\%;

    \item O algoritmo de aprendizado de máquina mais utilizado foi a árvore de decisão C4.5/J48 (34,78\%).
\end{itemize}

Considerando o foco da tese em transcodificação rápida para o formato AV1, realizamos uma análise de complexidade do seu software de referência, o \textit{libaom}. Mais explicitamente, avaliamos o custo computacional de processar a árvore de particionamento de blocos do \textit{libaom} e o seu impacto na eficiência de codificação. A análise demonstrou que, dependendo da combinação de profundidades que se permite ao \textit{libaom} utilizar, podemos reduzir a complexidade de execução desde 12,88\% (a um custo de 0,39\% de BD-rate) até 76,66\% (a um custo de 103,56\% de BD-rate).

As análises também permitiram observar a distribuição das profundidades da árvore de particionamento de blocos. Por consequência, evidenciamos dois fatos que podem auxiliar na aceleração da transcodificação de vídeo usando o AV1 como destino: 1) identificar previamente a necessidade do \textit{libaom} de aprofundar o nível de profundidade tende a otimizar o funcionamento do \textit{libaom} e 2) evitar que o \textit{libaom} considere testar profundidades maiores que o necessário tende a reduzir a complexidade de execução do \textit{libaom}.


A hipótese explorada nesta tese de doutorado foi a de que ``\textit{a transcodificação de vídeo para o formato AV1 pode ser acelerada, com baixo impacto na eficiência de codificação, por meio do uso de uma metodologia comum para treinamento de modelos preditivos baseados em aprendizado de máquina, buscando inferir decisões na codificação a partir de dados extraídos tanto do bitstream original como do próprio processo de re-codificação}''. Assim, as explorações realizadas tiveram por objetivo \textbf{o desenvolvimento de um \textit{pipeline} de processamento que possibilitasse a criação ágil de transcodificadores rápidos} a partir de diversos formatos de codificação de vídeo e com o AV1 como formato de destino. 

Como resultado, além do \textit{pipeline} desenvolvido, a tese apresenta sete propostas de transcodificação rápida para o formato AV1. Duas delas são baseadas em análises estatísticas e heurísticas para predição das profundidades ou de orientação de blocos do AV1, enquanto cinco são baseadas em modelos preditivos gerados por algoritmos de aprendizado de máquina.

A proposta do transcodificador rápido de \textbf{H.265/HEVC para AV1} baseado em \textbf{heurísticas} se amparou na observação dos tamanhos de \textit{Coding Units} (CU) extraídos durante a decodificação do vídeo. Além disso, propomos a utilização de limitadores inferiores e superiores da árvore de particionamento do \textit{libaom}, tendo como base a profundidade da CU observada para uma determinada região do vídeo. Obtivemos, deste trabalho, 25 combinações de limitadores e, consequentemente, 25 resultados de transcodificação. Assim, foi possível chegar a uma aceleração máxima da transcodificação de 61,20\% (a um custo de 14,66\% de BD-rate) usando um limitador inferior e superior rígido, ou seja, aplicando, no AV1, a mesma profundidade de árvore que foi observada no H.265/HEVC. A solução permite funcionamento configurável, levando a uma \textbf{aceleração média de 40\% (a um custo de 5,02\% de BD-rate)}, ou 32\% (a um custo de 6,16\% de BD-rate), ou 20\% (a um custo de 3,90\% de BD-rate), ou 15,10\% (a um custo de 3,51\% de BD-rate), ou mesmo de 5\% (a um custo de 0,11\% de BD-rate).

A proposta do transcodificador rápido de \textbf{VP9 para AV1} baseado em \textbf{heurísticas} visou identificar a melhor orientação de blocos a ser aplicado na codificação AV1. Essa identificação se amparou  na correlação entre o nível de profundidade observado na decodificação do VP9, no nível de profundidade em processamento na codificação do AV1 e no nível de quantização utilizado. Através dessas três variáveis, aplicamos a orientação recomendada: bloco quadrático, blocos retangulares orientados à horizontal ou blocos retangulares orientados à vertical. Com essa proposta, foi possível obter uma \textbf{aceleração de 28,16\%, com uma perda de eficiência de codificação em 4,34\%}.

O uso de modelos preditivos gerados por \textbf{algoritmo de aprendizado de máquina} para acelerar a transcodificação de \textbf{VP9 para AV1} foi pensado para tomada de decisão nos três primeiros níveis de profundidade do AV1: 0 (do bloco quadrático de 128$\times$128 para 64$\times$64), 1 (do bloco quadrático de 64$\times$64 para 32$\times$32) e 2 (do bloco quadrático de 32$\times$32 para 16$\times$16). A fim de aumentar a precisão da resposta obtida pelo modelo, treinamos um modelo para cada profundidade e para cada nível de quantização utilizado (20, 32, 43 e 55). Portanto, a proposta do transcodificador rápido de VP9 para AV1 usou 12 modelos gerados por árvore de decisão a fim de obter indicativo de particionamento ou não do bloco atual em processamento. Essa solução é capaz de apresentar uma \textbf{aceleração média de 16,76\% a um custo 5,10\% de BD-rate}.

Com base na proposta desenvolvida de VP9 para AV1, usando modelos de aprendizado de máquina, desenvolvemos um \textit{pipeline} de processamento que automatizou as etapas de treinamento, teste e predição dos modelos. Esse \textit{pipeline} de processamento, considerando metodologias bem definidas, aplica automaticamente a transcodificação para obtenção das amostras para treinamento e teste e as transcodificações originais e rápidas, a fim de obter resultados comparativos de eficiência de codificação e redução de complexidade. Esse \textit{pipeline} facilita a adaptação do algoritmo em novas propostas, o que possibilitou o desenvolvimento de outras quatro soluções de transcodificadores para o AV1 de forma ágil, sendo que três delas são inéditas na literatura científica conhecida.

A proposta de transcodificador rápido de \textbf{H.264/AVC para AV1} baseado em \textbf{modelos de aprendizado de máquina}, desenvolvido através do \textit{pipeline} proposto, é capaz de atingir \textbf{reduções de complexidades de 25,05\%, a um custo de 5,58\% de BD-rate}. Destacamos que essa é uma solução inédita na literatura científica.

A proposta de transcodificador rápido de \textbf{VP8 para AV1} baseado em \textbf{modelos de aprendizado de máquina}, também desenvolvido através do \textit{pipeline} proposto, é capaz de atingir \textbf{reduções de complexidades média de 55,69\%, a um custo elevar o BD-rate em 12,85\%}. Destacamos que essa também é uma solução inédita na literatura científica.

A proposta de transcodificador rápido de \textbf{H.265/HEVC para AV1} baseado em \textbf{modelos de aprendizado de máquina}, também desenvolvido através do \textit{pipeline} proposto, obtém uma \textbf{aceleração de 28,36\%, a um custo de 6,74\% de BD-rate}.

A proposta de transcodificador rápido de \textbf{H.266/AVC para AV1} baseado em \textbf{modelos de aprendizado de máquina} é capaz de obter uma \textbf{aceleração de 12,05\% do tempo de codificação a um impacto de 1,67\% no BD-rate}, ao ser desenvolvido utilizando o \textit{pipeline} proposto. Destacamos que essa é uma solução inédita na literatura científica e a que melhor apresentou resultados gerais dentre todas as propostas de transcodificações rápidas desenvolvidas nesta tese.

Por fim, além das contribuições com soluções de transcodificadores rápidos para o formato AV1 já publicados e descritos um a um no Apêndice \ref{apx:C}, destacamos a \textbf{expansão do conhecimento acadêmico na área de transcodificação de vídeo}, em especial através do conteúdo disponibilizado no capítulo \ref{cap:3} desta tese e das soluções de transcodificadores rápidos para o formato AV1 ainda não publicados. Além disso, o desenvolvimento de um \textit{pipeline} de processamento tem potencial para auxiliar os pesquisadores e desenvolvedores a obter soluções rápidas para diversas combinações de formatos.

