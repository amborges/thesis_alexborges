\section{Metodologia e Ferramental Utilizado}
\label{cap:7.1}

Apesar de os vídeos de ultra alta definição estarem crescendo em uso, principalmente UHD4K (de 3840$\times$2160 ou 4096$\times$2160 pixels), é a resolução de alta-definição 1080 (HD1080, de 1920$\times$1080 pixels) que é a mais consumida pelos usuários de serviços de \textit{streaming} \cite{bib:bitmovin_twitter}. Portanto, nas soluções desenvolvidas neste capítulo, dedicamos esforços para apresentar resultados nessa categoria de vídeos, mais especificamente em sequências de vídeo naturais HD1080. No entanto, sequências de vídeos naturais HD720 e UHD4K também serão consideradas, sempre que possível, para avaliação dos modelos preditivos. É importante destacar que, como qualquer outro trabalho envolvendo aprendizado de máquina, os conjuntos de dados precisam ser divididos em três subconjuntos, conforme discutido na seção \ref{cap:2.2}: de treinamento, de teste e de predição. Visando construir um vasto conjunto de vídeos, independentes entre si, utilizamos todos os vídeos HD1080 disponíveis nas condições comuns de teste dos documentos \cite{bib:av2_avm}, \citet{bib:hevcctc} e \citet{bib:ietfnetvct}, para compor, respectivamente, os conjuntos de treino (sete sequências), de teste (sete sequências) e de predição (demais sequências). Como pode ser visto na Tabela \ref{tab:XVII}, fazem parte dos vídeos de predição 48 sequências, distribuídas entre as resoluções HD720, HD1080 e UHD4K. Como foi dito no capítulo \ref{cap:4}, o Apêndice \ref{apx:A} descreve com detalhes as sequências utilizadas. Na Tabela \ref{tab:XVII}, constam os nomes das sequências utilizadas em cada uma das fases de desenvolvimento das soluções propostas neste capítulo.

\begin{table}
\begin{center}
\caption{Sequências de vídeo selecionadas para compor os conjuntos dos experimentos.}
\label{tab:XVII}
\footnotesize

\begin{tblr}{
    colspec = {l|p{13cm}},
    hlines,
    row{even} = {gray9}
}
\hline
\textbf{Fase} & \textbf{Sequências}  \\
Treinamento & BasketballDrive\_1920x1080\_50, BQTerrace\_1920x1080\_60, Cactus\_1920x1080\_50, CrowdRun\_1920x1080\_25, Kimono1\_1920x1080\_24, ParkScene\_1920x1080\_24, Tennis\_1920x1080\_24 \\
Teste        & FountainSky\_1920x1080p30\_130f, TimeLapseStreet\_1920x1080p30\_130f, Wheat\_1920x1080, WorldCup\_1920x1080\_30p, WorldCup\_far\_1920x1080\_30p, WorldCupFarSky\_1920x1080\_30p, Skater227\_1920x1080\_30fps \\
Predição    & aspen\_1080p\_60f, crowd\_run\_1080p50\_60f, dark720p\_120f, ducks\_take\_off\_1080p50\_60f, FourPeople\_1280x720\_60, FourPeople\_1280x720\_60\_120f, gipsrestat720p\_120f, Johnny\_1280x720\_60, Johnny\_1280x720\_60\_120f, KristenAndSara\_1280x720\_60, KristenAndSara\_1280x720\_60\_120f, Netflix\_Aerial\_1920x1080\_60fps\_8bit\_420\_60f, Netflix\_Boat\_1920x1080\_60fps\_8bit\_420\_60f, Netflix\_Crosswalk\_1920x1080\_60fps\_ 8bit\_420\_60f, Netflix\_DinnerScene\_1280x720\_60fps\_8bit\_420\_120f, Netflix\_DrivingPOV\_1280x720\_60fps\_8bit\_420\_120f, Netflix\_FoodMarket\_1920x1080\_60fps\_8bit\_420\_60f, Netflix\_FoodMarket2\_1280x720\_60fps\_8bit\_420\_120f, Netflix\_PierSeaside\_1920x1080\_60fps\_8bit\_420\_60f, Netflix\_RollerCoaster\_1280x720\_60fps\_8bit\_420\_120f, Netflix\_SquareAndTimelapse\_1920x1080\_60fps\_8bit\_420\_60f, Netflix\_Tango\_1280x720\_60fps\_8bit\_420\_120f, Netflix\_TunnelFlag\_1920x1080\_60fps\_8bit\_420\_60f, old\_town\_cross\_1080p50\_60f, park\_joy\_1080p50\_60f, pedestrian\_area\_1080p25\_60f, rush\_field\_cuts\_1080p\_60f, rush\_hour\_1080p25\_60f, station2\_1080p25\_60f, Vidyo1\_1280x720\_60, vidyo1\_720p\_60fps\_120f, Vidyo3\_1280x720\_60, vidyo3\_720p\_60fps\_120f, Vidyo4\_1280x720\_60, vidyo4\_720p\_60fps\_120f, boat\_hdr\_amazon\_720p, guitar\_hdr\_amazon\_1080p, pan\_hdr\_amazon\_1080p, rain\_hdr\_amazon\_720p, seaplane\_hdr\_amazon\_1080p, Netflix\_BarScene\_4096x2160\_60fps\_10bit\_420\_60f, Netflix\_BoxingPractice\_4096x2160\_60fps\_10bit\_420\_60f, Netflix\_Dancers\_4096x2160\_60fps\_10bit\_420\_60f, Netflix\_Narrator\_4096x2160\_60fps\_10bit\_420\_60f, Netflix\_RitualDance\_4096x2160\_60fps\_10bit\_420\_60f, Netflix\_ToddlerFountain\_4096x2160\_60fps\_10bit\_420\_60f, Netflix\_WindAndNature\_4096x2160\_60fps\_10bit\_420\_60f, street\_hdr\_amazon\_2160p \\
\hline
\end{tblr}
\end{center}
\end{table}


\subsection{Algoritmos de Aprendizado de Máquina}
\label{cap:7.1.3}

No capítulo \ref{cap:4}, especificamos que a linguagem Python versão 3 foi utilizada para o desenvolvimento de algumas soluções apresentadas nesta tese, em particular as apresentadas no atual capítulo. Dentre essas soluções, está o treinamento de modelos preditivos gerados por algoritmos de aprendizado de máquina. Existem diversas ferramentas, desenvolvidas em Python, que possibilitam essa geração de modelos preditivos, dentre elas o pacote \textit{Scikit-Learn} \cite{bib:scikitlearn-site}. Esse pacote oferece um algoritmo para treinamento de árvores de decisão que permitem a classificação e a regressão de valores, denominado \textit{Classification and Regression Trees} (CART) \cite{bib:scikitlearn_cart}, baseado no trabalho de \citet{bib:livroCART}. Este algoritmo permite a manipulação de alguns hiperparâmetros, descritos na Tabela \ref{tab:XVIII}. Nesta tabela apresentamos os hiperparâmetros existentes no algoritmo CART e seus respectivos valores usados nas soluções propostas de transcodificação rápida com uso de modelos gerados pelo algoritmo CART. Na última coluna da Tabela \ref{tab:XVIII}, apresentamos a quantidade de variações existentes para cada hiperparâmetro, considerando os valores que utilizaremos. Logo, calculando-se o produto dessas variações, há um total de 9,5 milhões de modelos candidatos de aprendizado de máquina a serem treinados com este algoritmo. Observe que, em vários hiperparâmetros da Tabela \ref{tab:XVIII} que utilizam valores inteiros como entrada, consideramos um valor máximo de 25. A razão disso é a quantidade de atributos utilizados nos trabalhos deste capítulo, que é de 25, conforme consta na seção \ref{cap:7.3}.

\afterpage{
\clearpage

\begin{landscape}
{\footnotesize
\begin{longtblr}[
    caption = {Relação dos hiperparâmetros disponíveis no algoritmo CART.},
    label = {tab:XVIII}
]{
    colspec = {l|p{14cm}|p{4.2cm}|r},
    rowhead = 1,
    hlines,
    row{even} = {gray9}
}
\hline
\textbf{Hiperparâmetro} & \textbf{Descrição} & \textbf{Valores Utilizados} & \textbf{Quantidade} \\
\textit{Criterion} & Define a função para medir a qualidade do subparticionamento árvore de decisão. Podem ser utilizados ganho de impureza gini ou ganho de informação por entropia de Shannon, conforme descritos em \citet{bib:CART_matematica}. & 'gini', 'entropy' & 2 \\
\textit{Splitter} & Define a estratégia utilizada para escolher o subparticionamento em cada nó da árvore de decisão. É possível optar pela melhor divisão ou divisão aleatória. & 'best', 'random' & 2 \\
\textit{Max Depth} & Define a profundidade máxima da árvore de decisão. O valor ‘None’ indica que não há um limite definido. & 'None', 3, 5, 7, 11, 13, 17, 19, 25 & 9 \\
\textit{Min Samples Split} & Define o número mínimo de amostras necessárias para subparticionar um nó interno (no mínimo duas amostras). & 2, 3, 5, 7, 11, 13, 17, 19, 25 & 9 \\
\textit{Min Samples Leaf} & Define o número mínimo de amostras necessárias para tornar um nó como folha. Ou seja, o subparticionamento em qualquer profundidade só será considerado se, pelo menos este número de amostras, estiver tanto no ramo do lado esquerdo como no lado direito.  & 1, 3, 5, 7, 11, 13, 17, 19, 25 & 9 \\
\textit{Max Features} & Define o número de atributos a serem considerados ao procurar o melhor subparticionamento do nó. Pode ser a raiz quadrada, ou o logaritmo de base dois da quantidade de atributos disponíveis, ou nenhum, indicando que todos os atributos devem ser utilizados. & 'sqrt', 'log2', 'None' & 3 \\
\textit{Max Leaf Nodes} & Define o número máximo de nós folhas. Em caso de ‘None’, não há limite máximo. A escolha pelos melhores nós é definida pelo próximo atributo. & 'None', 3, 5, 7, 11, 13, 17, 19, 25 & 9 \\
\textit{Min Impurity Decrease} & Define se um nó deve ser subparticionamento ou não, caso esta divisão induza a uma diminuição na impureza maior ou igual ao valor definido. & 0,0, 0,1, 0,2, 0,3, 0,4, 0,5, 0,6, 0,7, 0,8, 0,9, 1,0 & 11 \\
\textit{CPP Alpha} & Define o valor a ser usado para poda antecipada considerando a mínima taxa entre custo e complexidade (\textit{Minimal Cost-Complexity Pruning}, conforme \citet{bib:CART_cppalpha}). A subárvore com uma taxa inferior ao definido será escolhida. Se 0.0, nenhuma poda é executada. & 0,0, 0,1, 0,2, 0,3, 0,4, 0,5, 0,6, 0,7, 0,8, 0,9, 1,0 & 11 \\
\SetCell[c=3]{r}Quantidade Total de Modelos &&& 9.526.572 \\
\hline
\end{longtblr}
}
\end{landscape}
}


\subsection{Mensuração de Resultados}
\label{cap:7.1.4}

Na seção \ref{cap:2.4} discutimos as métricas estatísticas \textit{F1-Score} (Equação \ref{eq:6}) e AUC (Figura \ref{fig:8}) para avaliação dos treinamentos dos modelos preditivos gerados por algoritmos de aprendizado de máquina. O pacote \textit{scikit-learn} oferece meios de mensurar essas métricas, por meio do módulo ``\textit{sklearn.metrics}'', que permite utilizar as funções ``\textit{f1\_score}'' \cite{bib:scikitlearn_f1} e ``\textit{roc\_auc\_score}'' \cite{bib:scikitlearn_auc}. Essas duas funções possuem diversos parâmetros de utilização, conforme pode ser observado em documentação própria \cite{bib:scikitlearn_f1, bib:scikitlearn_auc}. Serão utilizados os valores padrões dessas funções, exceto por um único parâmetro: ``\textit{average}''. Por padrão, esse parâmetro avalia apenas os resultados positivos preditos pelo modelo treinado; no entanto, apesar dos modelos treinados pelas nossas propostas terem rótulos binários (ver mais na seção \ref{cap:7.3}), a relevância das respostas positivas tanto quanto as negativas possuem igual importância. Dessa forma, como é importante que as funções ``\textit{f1\_score}'' e ``\textit{roc\_auc\_score}'' avaliem ambas respostas do rótulo igualmente, o parâmetro ``\textit{average}'' deve ser configurado como ``\texttt{macro}''.

Em relação às métricas para comparação dos resultados de transcodificação rápida, empregamos os valores de TS (Equação \ref{eq:7}) e de BD-rate (Figura \ref{fig:9}), ambos já apresentados na seção \ref{cap:2.5}. Em relação à captura do tempo de processamento, valor importante para gerar o TS, usamos os valores informados pelo próprio software de referência do AV1, cujo tempo total da codificação é mensurado em milissegundos, posteriormente convertidos para segundos. Já no caso do BD-rate, utilizamos os valores de bitrate (em kilobits por segundo, kbps) e de PSNR-Y (em decibéis, dB).
