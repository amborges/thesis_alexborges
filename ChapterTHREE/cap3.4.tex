\section{Considerações Finais Sobre o Estado da Arte}
\label{cap:3.4}

Nesta revisão sistemática da literatura, apresentamos soluções publicadas visando a aceleração da transcodificação de vídeo. Como a análise de toda a literatura científica é impraticável, definimos, no início do capítulo \ref{cap:3}, a metodologia de busca aplicada, o que nos permitiu selecionar os trabalhos mais relevantes publicados com foco na aceleração da transcodificação de vídeo. Dessa forma, de 1500 trabalhos iniciais, selecionamos um total de 34 artigos que focaram especificamente no assunto de transcodificação acelerada e que apresentaram análise dos resultados em termos de redução da complexidade e eficiência de codificação. 

Através da revisão sistemática da literatura, concluímos que a maioria dos trabalhos publicados propõe soluções para padrões oriundos da família ITU-VCEG (por exemplo, H.264/AVC, H.265/HEVC e H.266/VVC), representando 100\% de todas as transcodificações de vídeo homogênea. Por outro lado, nas transcodificações de vídeo heterogêneas, esses padrões da ITU-VCEG estão presentes em 88,46\% dos formatos de origem e em 76,92\% dos formatos de destino. Além disso, cerca de 61\% dos trabalhos de transcodificação heterogênea são de H.264/AVC-para-H.265/HEVC. E, dentre todos os trabalhos selecionados e revisados, é possível concluir que a maioria das soluções emprega algum tipo de herança de particionamento de blocos para acelerar o processo de transcodificação, estando presente em 70\% dos casos. Por essa razão, no capítulo \ref{cap:5}, discutiremos em maiores detalhes as estruturas de particionamento do formato AV1 e analisaremos o seu impacto no processo de codificação.

Em média, as soluções revisadas atingem uma redução do tempo de transcodificação de aproximadamente 50,74\%, em comparação com um transcodificador original. Essa aceleração é alcançada a um custo médio de 4,11\% em perdas de eficiência de codificação. A proposta que apresentou a maior redução de complexidade foi a de \citet{bib:leuven_2011}, com um TS igual à 95,73\%. Por outro lado, a proposta de \citet{bib:aminlou_2016} apresentou o menor TS, com modestos 6,80\% de aceleração em comparação com o transcodificador original. A melhor solução em termos de eficiência de codificação foi a de \citet{bib:grellert_2018}, cujo trabalho impactou o BD-rate em apenas 0,29\%. Contrastando com esse resultado, \citet{bib:zhang_2012} apresentou um transcodificador rápido que gera um acréscimo de 30\% em BD-rate. Uma análise mais detalhada dos resultados concluiu que as soluções baseadas em aprendizado de máquina alcançam um equilíbrio muito melhor entre aceleração e eficiência de codificação, com valores de BD-rate geralmente abaixo de 1\% e TS de até 70\%.

Embora diferentes algoritmos de aprendizado de máquina tenham sido empregados ao acelerar a transcodificação de vídeo, o algoritmo de árvore de decisão C4.5/J48 foi a escolha principal dos trabalhos, representando 34,78\% dos casos. Uma análise sobre todas as soluções baseadas em aprendizado de máquina revelou que a média e a variação de blocos residuais, as informações baseadas no vetor de movimento e as informações de tamanho de bloco foram os atributos mais comumente selecionados para treinar os modelos de ML. Em 93\% dos artigos revisados, as decisões rápidas propostas focaram na escolha de estruturas de particionamento de blocos, seja eliminando possibilidades de tamanhos e/ou formatos de blocos ou encerrando antecipadamente o processo de particionamento. Não identificamos nenhuma relação entre o número de atributos usados para treinar os modelos e a precisão do modelo ou desempenho do transcodificador em termos de BD-rate ou TS, já que a literatura revisada inclui soluções que empregam desde apenas três atributos \cite{bib:chen_2019, bib:wei_2017} até quase 300 \cite{bib:fernandez2_2006, bib:holder_2009}. Portanto, é aconselhável que pesquisadores, investigando novas soluções baseadas em aprendizado de máquina, alimentem os algoritmos com o maior número possível de atributos, deixando a cargo do próprio modelo decidir quais deles deverão ser considerados.

Esta pesquisa permitiu identificar problemas de transcodificação ainda inexplorados e a falta de soluções para alguns transcodificadores específicos. Por exemplo, nenhuma solução foi identificada para acelerar a transcodificação oriunda de ou com destino para formatos de vídeo chineses, exceto para o transcodificador AVS-para-H.264/AVC proposto por \citet{bib:jin_2011}. Não identificamos, por exemplo, soluções para AV1-para-AVS3, AVS3-para-H.266/VVC ou H.265/HEVC-para-AVS2. A ausência de formatos chineses na literatura é algo que chama a atenção, já que a China representa aproximadamente 18\% de todos os potenciais consumidores mundiais \cite{bib:world_population}. Além disso, o número significativamente alto de novos modos decisórios em novos formatos ainda é inexplorado em soluções de transcodificação. Por exemplo, AV1 e H.266/VVC incluem um complexo sistema preditivo, tanto intraquadro como interquadros, com vários novos modos que exigem um grande esforço computacional para serem testados. Portanto, são esperadas soluções com foco na herança de informações oriundas de outras partes do \textit{bitstream} decodificado para acelerar as predições nesses novos codecs.

A análise do estado da arte em transcodificação rápida de vídeo permitiu, entre outras coisas, definir o processo de particionamento como foco das estratégias investigadas e apresentadas nesta tese. Assim, tendo como base a metodologia geral utilizada na pesquisa apresentada nesta tese (capítulo \ref{cap:4}), o capítulo \ref{cap:5} apresenta uma análise da complexidade do processo de particionamento do formato AV1, expandindo o conteúdo desenvolvido na seção \ref{cap:3.3}.
